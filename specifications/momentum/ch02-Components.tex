\section{List of data set components}

% Generic list of metadata and data fields we use for ll file formats. Don't repeat full UGRID conventions, just state what we use and our conventions, and any extensions we use

% Note this is taken from the current output from the NWP suite. This applies some post-processing of metadata and data  to the raw model output

This section lists all variables and attributes that are used in the Momentum UGRID-NetCDF file formats with a brief explanation of what they are used for. Refer to the CF and UGRID conventions for more information.

\subsection{Global Attributes}

These are metadata that the model writes as standard and generally not changed.

% Do we need other things in here - e.g. that the output is from the Met Office Momentum model?

\begin{verbatim}
description
interval_write
title
Conventions
\end{verbatim}

\subsection{Dimensions}

Each NetCDF dataset is characterised by a set of dimensions which contains the full range of sizes and 
shapes necessary to describe the data of every variable in the dataset. For each 2D mesh present in the dataset there are three dimensions which correspond to the number of nodes, edges and faces on the mesh. 

.Standard Topology dimensions
\begin{verbatim}
nMesh_id_node
nMesh_id_edge
nMesh_id_face
Mesh_id_face_N_nodes
Mesh_id_edge_N_nodes
Mesh_id_face_N_edges
Mesh_id_face_N_faces
Mesh_id_edge_N_faces
\end{verbatim}

Standard Time Dimensions

Time is represented as an umlimited NetCDF dimension, meaning that new data can be appended easily to that dimension for time-series capability. 

% check unlimited, is this necessary for HDF5 encoded files?

\begin{verbatim}
time_instant
time_centered
\end{verbatim}

% Does this imply time averages will be labelled with the central time? This is I think a departure from um/pp approach. Perhaps a comment would be useful.

Other Miscellaneous Dimensions

There are application-specific dimensions of several kinds. Vertical dimensions (e.g. model levels, pressure levels,
depth); Non-spatial dimensions (e.g. tiles or pseudo-levels)

Finally, there are some additional dimensions: bnds which contains the number of bounds for a 1D quantity (i.e. 2) and dim1 for the number of bounds for a face (i.e. 4).

% These are named horribly - maybe they could be changed to something more informative?

\begin{verbatim}
boundary_layer_type
bnds
dim1
depth
pressure
\end{verbatim}

\subsection{Mesh and Topology Variables and Attributes}

These use standard topology dimensions as defined in section 2.2.

\begin{verbatim}
int     Mesh_id
    Mesh_id:cf_role
    Mesh_id:topology_dimension
    Mesh_id:long_name
    Mesh_id:node_coordinates
    Mesh_id:edge_coordinates
    Mesh_id:face_coordinates
    Mesh_id:face_node_connectivity
    Mesh_id:edge_node_connectivity
    Mesh_id:face_edge_connectivity
    Mesh_id:face_face_connectivity
    Mesh_id:edge_face_connectivity
\end{verbatim}

\begin{verbatim}
float Mesh_id_node_x(nMesh_id_node) 
    Mesh_id_node_x:units
    Mesh_id_node_x:standard_name
    Mesh_id_node_x:long_name
float Mesh_id_node_y(nMesh_id_node)
    Mesh_id_node_y:units
    Mesh_id_node_y:standard_name
    Mesh_id_node_y:long_name
float Mesh_id_edge_x(nMesh_id_edge)
    Mesh_id_edge_x:units
    Mesh_id_edge_x:standard_name
    Mesh_id_edge_x:long_name
float Mesh_id_edge_y(nMesh_id_edge)
    Mesh_id_edge_y:units
    Mesh_id_edge_y:standard_name
    Mesh_id_edge_y:long_name
float Mesh_id_face_x(nMesh_id_face) 
    Mesh_id_face_x:units
    Mesh_id_face_x:standard_name
    Mesh_id_face_x:long_name
float Mesh_id_face_y(nMesh_id_face)
    Mesh_id_face_y:units
    Mesh_id_face_y:standard_name
    Mesh_id_face_y:long_name
int Mesh_id_face_nodes(nMesh_id_face, Mesh_id_face_N_nodes) 
    Mesh_id_face_nodes:long_name
    Mesh_id_face_nodes:cf_role
    Mesh_id_face_nodes:start_index
int Mesh_id_edge_nodes(nMesh_id_edge, Mesh_id_edge_N_nodes)
    Mesh_id_edge_nodes:long_name
    Mesh_id_edge_nodes:cf_role
    Mesh_id_edge_nodes:start_index
int Mesh_id_face_edges(nMesh_id_face, Mesh_id_face_N_edges)
    Mesh_id_face_edges:long_name
    Mesh_id_face_edges:cf_role
    Mesh_id_face_edges:start_index
int Mesh_id_face_links(nMesh_id_face, Mesh_id_face_N_faces)
    Mesh_id_face_links:long_name
    Mesh_id_face_links:flag_meanings
    Mesh_id_face_links:flag_values
    Mesh_id_face_links:cf_role
    Mesh_id_face_links:start_index
int Mesh_id_edge_face_links(nMesh_id_edge, Mesh_id_edge_N_faces) 
    Mesh_id_edge_face_links:long_name
    Mesh_id_edge_face_links:comment
    Mesh_id_edge_face_links:cf_role
    Mesh_id_edge_face_links:start_index
\end{verbatim}

\subsection{Time Variables and Attributes}

% Validity time required?

These use standard time dimensions as defined in section 2.2.

\begin{verbatim}
double time_instant(time_instant)
    time_instant:axis
    time_instant:units
    time_instant:standard_name
    time_instant:long_name
    time_instant:calendar
    time_instant:time_origin

int64 forecast_period(time_instant)
    forecast_period:units
    forecast_period:standard_name

int64 forecast_reference_time
    forecast_reference_time:units
    forecast_reference_time:standard_name
    forecast_reference_time:calendar

double time_centered(time_centered) ;
    time_centered:axis
    time_centered:bounds
    time_centered:units
    time_centered:standard_name
    time_centered:long_name
    time_centered:calendar
    time_centered:time_origin

double time_centered_bnds(time_centered, bnds)

int64 forecast_period_0(time_centred)
    forecast_period_0:bounds
    forecast_period_0:units
    forecast_period_0:standard_name

int64 forecast_period_0_bnds(time_centered, bnds)
\end{verbatim}
% I do not know what this is. If it is standard it should be named more informatively

\subsection{Data Variables and Attributes}

These use standard topology and time dimensions as well as miscellaneous dimensions as 
defined in section 2.2. 

\begin{verbatim}
double data_var_surface([time dim], [mesh dim]) ;
    data_var_surface:long_name
    data_var_surface:units
    data_var_surface:interval_operation
    data_var_surface:online_operation
    data_var_surface:cell_methods
    data_var_surface:mesh
    data_var_surface:location
    data_var_surface:coordinates


type data_var_layer([time dim], [layer dim], [mesh dim]) ;
    data_var_layer:long_name
    data_var_layer:units
    data_var_layer:interval_operation
    data_var_layer:online_operation
    data_var_layer:cell_methods
    data_var_layer:mesh
    data_var_layer:location
    data_var_layer:coordinates

type data_var_vector([dim]):
    data_var_vector:axis
    data_var_vector:bounds
    data_var_vector:units
    data_var_vector:long_name
    data_var_vector:positive

type data_var_vector_bnds([dim], [dim])


type data_var_scalar ;
    scalar_var:units
    scalar_var:long_name
\end{verbatim}


\subsection{Constants} 

% I don't think this is needed as much of it covered my metadata dimensions. Might be needed for any data constants we use?

\subsubsection{Integer}
\subsubsection{Real}
\subsubsection{Level Dependent}
\subsubsection{Row Dependent}
\subsubsection{Column Dependent}


\subsection{ADDITIONAL PARAMETERS}

% Is this needed?

This section was formerly known as field-dependent constants.
The data is stored as a 1D array. There are two dimension values, only the first of these is used. This section is used to store stochastic physics coefficient arrays.

% enumerate list including section numbering, e.g. 2.7.1

\begin{enumerate}
\item coeffc: spherical harmonic cosine coefficients for SPT scheme. size=stph n2 * (stph n2+1)
\item coeffs: spherical harmonic sine coefficients for SPT scheme. size=stph n2 * (stph n2+1)
\item dpsidtc: 2D version of d(psi)/d(t) COS coeffs in Fourier for SKEB scheme. size=(stph n2-stph n1) * (stph - n2+1)
\item dpsidts: 2D version of d(psi)/d(t) SIN coeffs in Fourier. size=(stph n2-stph n1) * (stph n2+1)
\item rp coef: Array of parameters for RP scheme. size=rp max (currently 25)
\end{enumerate}